\begin{abstract}
{\footnotesize
Advances in remote sensing technologies, data collection methods and data sharing platforms have led to an unprecedented increase in the abundance, availability and spatio-temporal resolution of ecological data over the past decade. This wealth of data has significantly benefited data-driven machine learning approaches, including deep learning, which can efficiently leverage these data to estimate current species distributions. However, deep learning models often struggle to extrapolate to future projections and novel conditions, limiting their relevance for forecasting future biodiversity changes under global changes.
% 
Alternatively, process-based models, which incorporate by-design constraints representing scientifically established knowledge on ecological mechanisms, are often thought to offer a better extrapolation potential. Yet, they are difficult to calibrate using data on species observations and heterogeneous sensors, meaning that they fail to capitalize on the recent increase of available data and often yield imprecise predictions in practice. 
% 
Scientific machine learning (SciML) approaches, which seek to integrate domain knowledge with machine learning methods, present a promising pathway to bridge this gap between data-driven and process-based modeling paradigms in ecology.
% 
This review examines the potential of SciML to improve biodiversity forecasting under global change. We (i) review objectives and unsolved challenges of forecasting species distributions; (ii) synthesize the strengths and limits of existing process-based and data-driven modeling approaches, (iii) identify invariant properties of ecological systems used in process-based modeling to guide the design of future SciML SDMs, acknowledging the process knowledge gaps that could be filled from data, and (iv) what machine learning can contribute to SciML SDM in terms of flexibility, scalability and data assimilation. Building on this comparison, we highlight differentiable programming as a fundamental principle enabling hybrid SciML models that integrate mechanistic processes with flexible learning architectures. We then survey existing efforts in hybrid ecological modeling, drawing connections to parallel advances in other scientific fields. Finally, we outline future directions for SciML in ecological forecasting.
% 
Our goal is to stimulate the development of more robust and interpretable species distribution models that can better support conservation decision-making in an era of rapid environmental change.
\vspace{1cm}

\noindent \textbf{Key words.} Scientific machine learning, hybrid models, deep species distribution models, predictive ecology

\noindent \textbf{Potential journals.} Methods in Ecology and Evolution, Ecological informatics, PLOS Computational Biology, Ecology letters, TREE, Nature Methods, Science Advances
}
\end{abstract}


\section{Squeleton}

\subsection{Objectives and challenges in forecasting biodiv in space and time} % Sara

Broad objectives of biodiversity models
- Learn eco-evolutionary processes governing species distributions (improving ecological knowledge):
\begin{itemize}
    \item Attribution to environmental processes: climate, land use, etc.
    \item Quantify the contribution of different ecological processes: trait-environment and/or phylogenetic effects, biotic interactions, dispersal/connectivity processes
\end{itemize}

- Improve species distribution  estimation (dealing with data challenges):
\begin{itemize}
    \item Integrate multi-source data: presence-only, presence absence, abundance
    \item Upscaling biodiversity data / downscaling environmental drivers
\end{itemize}

- Forecast species distributions under new conditions (in space and time):
\begin{itemize}
    \item Near-term forecasts
    \item Long-term projections
    \item Predict range shifts and colonization/extinction
\end{itemize}

- Support biodiversity monitoring, conservation adn restoration:
\begin{itemize}
    \item Inform sampling, guide efficient monitoring 
    \item Detect changes and turnover in assemblages and attribute to drivers    
    \item Identify early-warning signals and tipping points 
    \item Simulate biodiversity under different management scenarios (counterfactual simulations)
    \item Enable optimization of conservation strategies (models that integrate seamlessly)    
\end{itemize}

What current limits towards building more reliable, interpretable and usable predictions of biodiversity in space and time? Including future projections, but not only 
- Equilibrium hypotheses assumed by these models are not satisfied \\
- Biases and weakly labeled data \\
- Scale mismatch between data and processes  \\
- Lack of consideration of transient dynamics: seasonal and intra-annual processes are ignored by statistical models which can lead to erroneous future predictions that ignore transient dynamics  \\
- Lack of integration of dispersal and spatial connectivity: mostly done separately using IBMs or simplified with kernels 
- Overlooking biotic interactions and overall community context  \\
- Overconfident or misleading predictions due to unaccounted for sampling / model structure errors  \\
- Inadequacy for near-term forecasting, early warning signaling and tipping point detection  \\
- Difficulties to quantify uncertainty at all levels (about inputs like climate, ecological model, data, etc)  \\

Some points
- Difficulties to scale spatially, taxonomically
- Difficulties to attribute predictions to ecological processes coherently with scientific knowledge
- Integrating traits to pull predictive strength across taxa (e.g. Fate or Chalmandrier) -> predictive ability currently limited  (This is more a solution than an objective / challenge )


\subsection{Strength and advantages of process-based vs data driven models} % Diego

Machine learning models, and particularly those based on deep learning, are known to perform well on data that follow the same distribution seen during training, but to break down under distribution shifts, with the different failure models being still under research~\cite{nagarajan2020understanding}.
Despite this, it has also been shown that methodological developments that provide an improved in-distribution performance also tend to result in better out-of-distribution (OOD) capabilities~\cite{miller2021accuracy}, although the gap between both performances tends to widen as performance increases. 
This suggests that state-of-the-art deep learning approaches should not be discarded when seeking better OOD performance, though an increasing effort may be required to close the OOD gap.
Even on the in-distribution setting, deep learning-based SDMs tend to outperform traditional data-driven approaches, such as Random Forests, only when using complex and structured input variables, such as remote sensing images~\cite{deneu2021convolutional}, rather than rasters of environmental variables~\cite{kellenberger2024performance}.
In classical data-driven SDM models, it has been observed that higher model complexity tends to result in worse extrapolation under climate change projections~\cite{brun2020model}.
Despite this, data-driven models are often used to obtain future projections of species distributions~\cite{brun2016predictive,harris2018forecasting,brodie2022recommendations,barnes2022climate}.

%-Purely data driven models: predict well observed data, but unable to disentangle underlying ecological processes (or sampling biases ;-)) -> fail in extrapolation. 
%Yet, Currently many studies doing future projections with broad taxonomic spectrum are purely data-driven (e.g. static sdm), easier but unreliable "out-of-sample" e.g. for future projections

Process-based SDMs, those that explicitly include physical and biological processes in their computation, are favored over data-driven SDMs due to their better behavior under extrapolation~\cite{urban2016improving,briscoe2019forecasting}.
However, the data required to calibrate their parameters is much less readily available that the data used by correlative, data-driven, models: it turns out to be easier to scale the number of species occurrences needed to fit a data-driven model than to explicitly measure the parameters describing a biological process~\cite{urban2016improving}.
%-Purely process-based models (to define, incorporte scientific knowledge by-design constraining predictions?) appealing for more reliable extrapolation (e.g. some already used like dynamic vegetation models, physiological / phenological models of survival cf PHENOFIT, matrix population models, reaction diffusion models, etc) 
%Assumed to be more robust for future projection, see e.g. https://www.science.org/doi/10.1126/science.aad8466
%Also, provide interpretation for predictions making scientific sense

%Ideally, one calibrates each parameter based on specific data, BUT Impossible to scale to multi/many taxa (Often calibrated based on blurry / subjective expert knowledge) 

%Is Hybrid / inverse calibration of process-based models on observation a solution?
[Move this to later on. Here focus on on the two paradigm] 
A potential way forward is to use SciML approaches in order to calibrate parameters using the same scalable observations used by classical ML methods.
By using differentiable programming, it becomes possible to learn a set of parameters in a process-based or hybrid model that lead to predictions that match the observations~\cite{karniadakis2021physics,shen2023differentiable}.

- Rigid and wrong specification of model components lead to irrealistic parameter values when calibrated on observation data 
- Algorithms for inverse calibration (e.g. often Bayesian algorithms) come with much convergence problems 
- To avoid it, process complexity sacrified -> predictions biased 

\subsection{What can process-based model bring? process modeling and knowledge gap} % Christophe
Data driven approaches can potentially learn arbitrary relationships from the data. Yet, this can lead to unrealistic artifacts which can be revealed under extrapolation conditions.

Ecological processes knowledge enable to set some universal prior constraints on these relationships, which both act as a form of regularization (enabling to learn more robust models from less data) and a guarantee that predictions will remain credible outside of the observed data range.

We'll now (i) summarize invariant properties of ecological systems that have been embedded within existing process-based models to guide the design of scientific ML SDMs, (ii) point out the limits of process-based models currently used in ecology, (iiii) highlight gaps in process knowledge that could potentially be learnt from data. 

\paragraph{Invariant ecological properties in process-based models.} 

From niche theory, (i) limited species tolerances to abiotic conditions (Grinnell) + niche conservatism in time -> a species can only persist in a restricted range of physical conditions and this is invariant on reasonable time scales (say <10K years), (ii) limited tolerances to biotic conditions-> presence or absence of other species delimit a biotic sub-space where a species can persist, i.e. the Eltonian niche, (iii) a species is limited by its dispersal capabilities which are also constant in time, even though spread rates themselves will depend on local dispersal pathways. Dispersal crucial to explain persistence in presence of source-think dynamics, such as described by meta population dynamics (fragmented patches) or range dynamics (continuous suitable space). 

Ecological process invariance across species is often mediated by their traits (life-history, functional traits) which as they may explain how a species interact with others, reproduce in time, and move.

For instance, spatial dispersal is mediated by spatial connectivity which depends on landscape + environmental conditions but also on species traits. This dependence is often invariant across large groups of species (e.g. all plant species whose seeds are spread by wind).

Traits also important to determine how biotic conditions affect a species persistence, hence used to model interspecific interactions (e.g. \cite{chalmandrier2022predictions}).

Process based models are most often hypothesing Markovian transitions

Impact of disturbances natural (e.g. fires, flooding) or anthropogenic (e.g., agriculture, pollution, land use change) 

-> non-equilibrium / transient dynamics

Knowledge about variability also matters. For instance, we know that small and isolated populations are sensitive to demographic stochasticity and genetic drift <-> collapse or speciation (neutral theory) making these regimes less predictible. Fitting process-based models requires to account for how variance of demographic and spread rates scales in response to population size, life-stages or environmental gradient, otherwise we would, e.g. wrongly assume too much information on the growth rate value in the data of small population. 

\paragraph{Limits of process-based models used in ecology.} 

Due to the peculiarity and sensitivity of the cost functions to optimize (XXX), classic sampling (e.g. MCMC) or gradient descent algorithms often fail or require very specific tuning of their hyper-parameters to converge.

This is sometimes due to the lack of identifiability of the model parameters, a problem also called parameter-redundancy (\cite{cole2020parameter}), meaning the existence of distinct sets of parameters with equally explanatory power even with infinite data, and leading to ridges in the optimal surface of the optimized cost functions (e.g., negative log-likelihood). 

This can be also partly due to the rigidity of the functional relationships set a priori based on somehow arbitrary rules (XXX). 

As a consequence, methods to calibrate process-based models (aka inverse modeling) are often specific, like differential evolution MCMC algorithms or particle filters / sequential Monte-Carlo algorithms typically promoted for state-space models.

Even though such specific methods have been developped they  

Inability to deal with complexity -> Process-based models designed with limited number of components rigid functional components to minimize their 

Algorithms commonly used to calibrate , both Bayesian or based on minizing an observed 

\paragraph{Gaps in process knowledge}

Just some examples.

The functional form of growth rate, e.g. Ricker vs Beverton Holt vs Hassell etc, its dependence on the environment, on the presence of other species, on the existence of allee-effect, is highly context-dependent on the biology and ecology of the taxon, and thus context-dependent. -> Would be relevant to learn it from data

How traits mediate responses of dispersal, in particular long distance, and the effect of biotic interactions of growth is a big unknown (to me).


\subsection{What ML can bring?} % Victor

\subsubsection{Improving process-based model expressive power.}
\cite{wood2001}: ML to avoid irrelevant incidental assumptions on exact functional forms.

\subsubsection{Interface with different data sources}

\paragraph{Integrating multimodal data sources (encoder).}
\begin{itemize}
    \item within a multispecies context
    \begin{itemize}
        \item Leveraging trait data: using traits to infer interactions (trait-matching \cite{pichler2020machine}), using traits to constrain species-environmental relationships \cite{pollock2012role}, using traits to parameterize species interactions, dispersal and growth rates in process-based SDMs \cite{chalmandrier2022predictions}, using community-level traits to constrain SDM predictions (Deschamps et al, in revision).  
        \item Leveraging phylogenetic relationships
    \end{itemize}
    \item automatic feature extraction from structured data (computer vision) to parametrize certain processes
    \item Use of geo embedding for predictions (AlphaEarth)
\end{itemize}

\paragraph{Integrating multimodal data sources (for calibration, decoder)}
\begin{itemize}
    \item Formulation of loss functions integrating different datasources \cite{Schneider2017}
    \begin{itemize}
        \item citizen science data \cite{brun2024, gillespie2024}
        \item environmental DNA \cite{Ruppert2019}
        \item process-based predictions matched against e.g. satellite imagery with an encoder-decoder, 
        \item integration of bioaccoustics \cite{Aide2013}
    \end{itemize}
    
\end{itemize}

\subsection{A review of hybrid modelling approaches}

\subsection{Methods}



% LEGACY

\section{Introduction}
\noindent\textbf{Motivation.}
The current biodiversity crisis poses significant threats to humanity by negatively impacting ecosystem services \cite{Barnosky2011,ipbes2019}. Climate change and habitat degradations are major drivers of biodiversity loss, influencing species abundance, community composition, and spatial distribution \cite{ipbes2019, pereira2010}. Some species adapt to climate change by shifting their geographic range to track suitable environmental conditions \cite{bellard2012} when sufficient habitat connectivity supports colonization and survival \cite{hodgson2016}. However, anthropogenic pressures and land-use changes often result in habitat fragmentation, impeding species' ability to track environmental shifts effectively \cite{mcguire2016}. Realistic biodiversity forecasts are essential for guiding conservation actions that promote ecosystem resilience \cite{urban2015}. 

\noindent\textbf{What is species distribution modelling?}
Biodiversity forecasts typically rely on species distribution models (SDMs) that assume species are in equilibrium with their environment \cite{guisan2000}. These models typically overlook dynamical processes determining species’ range shifts and responses to environmental change, such as dispersal and demographic changes \cite{svenning2013}. Alternative models explicitly representing these processes, known as process-based models, are entailed with large inaccuracies, leading to imprecise forecasts in practice. Consequently, there is an urgent need to incorporate data-driven ecological dynamics into biodiversity models \cite{urban2015}. 

\noindent\textbf{What are current challenges?}
\begin{itemize}
    \item Often only a function of abiotic environmental variables is considered.  
    \item Deep learning models are bad at extrapolation, such as to new regions or under climate change.
    \item Interpretability, no way of verifying if the learned functions make biophysical/ecological sense.
\end{itemize}

\noindent\textbf{How could we move forward?}
\begin{itemize}
\item How to deal with the effect of other ecological processes (biotic interactions, dispersal history, evolutionary processes, meta-community dynamics etc.)?
\item Can we use existing knowledge to constrain a deep SDM in order to improve extrapolation? For instance, species-trait-environment knowledge could be used to build a concept bottleneck~\cite{koh2020concept}. 
\item Can this encoded knowledge allow assessing the plausibility of the data-driven relations learned by the model?
\end{itemize}

\noindent\textbf{What is scientific machine learning?}
Scientific machine learning** (SciML) is an emerging discipline within the data science community.  SciML seeks to address domain-specic data challenges and extract insights from scientific data sets through innovative methodological solutions \cite{rackauckas2020}. SciML draws on tools from both machine learning and scientific computing to develop new methods for scalable, domain-aware, robust, reliable, and interpretable learning and data analysis, and will be critical in driving the next wave of data-driven scientific discovery in the physical and engineering sciences. 

\noindent\textbf{What is the goal of this paper?} 
\begin{itemize}
    \item Explore opportunities offered by scientific ML approaches that could help deep SDMs have better extrapolation behaviour by leveraging biophysical/ecological knowledge.
    \item Explore types of knowledge that is, or could potentially be, available and useful for this.
    \item Identify use cases that could be used to benchmark extrapolation capacity and interpretability.
\end{itemize}

\section{Literature review on start-of-the-art species distribution models}

\subsection{Data-driven species distribution models}
\noindent\textbf{Classical SDMs}
 establish statistical relationships between species occurrence or abundance and environmental covariates, under the assumption that species–environment relationships are in equilibrium and that key environmental drivers are accurately measured and unbiased. 

Initially developed as environmental envelope models that delineated suitable conditions based on species’ known occurrences, Species distribution models (SDMs) have since evolved into statistically grounded tools that model species–environment relationships using presence, absence, or abundance data. 

Statistically, a species distribution modeling can be treated as a logistic regression or a binary classification problem and implemented via a variety of statistical and machine learning methods \cite{guisan2017habitat}, including: regression based approaches (e.g. Generalized Linear Models (GLM) \cite{nelder1972generalized}, Generalized Additive Models (GAM) \cite{hastie1986generalized}, Multivariate Adaptive Regression
Splines (MARS) \cite{friedman1991multivariate}), machine learning and classification methods (Support Vector
Machines (SVM) \cite{hearst1998support}, Artificial Neural Networks (NN) \cite{lecun2015deep}, Classification
and Regression Trees (CART) \cite{breiman2017classification}, boosting (e.g Boosted Regression Trees (BRT) \cite{elith2008working}) and bagging approaches (e.g Random Forests (RF) \cite{breiman2001random}), Maximum ENTropy \cite{phillips2017opening} in addition to ensemble models \cite{thuiller2009biomod}. 

User-friendly software packages (e.g., biomod2 \cite{thuiller2009biomod}, dismo \cite{hijmans2017package}, ENMtools \cite{warren2010enmtools}) have greatly expanded accessibility of SDMs in ecological research. Furthermore, the community has moved toward increased standardization with the proposal of standardized protocols for model building, evaluation \cite{roberts2017cross,valavi2018blockcv} and reporting \cite{fitzpatrick2021odmap,zurell2020standard}. 

%% maybe add a few words on presence/only questions and incorporating observation processes with occupancy models ??

\noindent\textbf{Joint SDMs}
Joint Species Distribution Models (JSDMs) \cite{pollock2014understanding} represent an extension of SDMs that allow for the simultaneous modeling of multiple species’ distributions. These models account not only for species–environment relationships but also for residual correlations that may reflect biotic associations.

Recent advances in JSDMs have driven the development of various statistical frameworks and computational tools. Initial implementations were grounded in Generalized Linear Mixed Models (GLMMs) \cite{pollock2014understanding,ovaskainen2020joint,warton2015so}, where fixed effects represent species’ responses to environmental covariates, and random effects capture residual interspecific correlations. The primary methodological differences among JSDMs can be attributed to three main modeling choices \cite{wilkinson2019comparison}: (1) the use of hierarchical structures for regression coefficients, (2) the modeling of residual covariance either directly or through latent variables, and (3) the application of dimension reduction techniques.

In practice, the marginal predictions of JSDMs are often no better than those from single GLMs \cite{poggiato2021interpretations} and are typically outperformed by machine learning methods \cite{norberg2019comprehensive}. However, their main advantage lies in conditional predictions, i-e estimating the presence of one species given others \cite{wilkinson2021defining}. Finally, JSDMs do not directly model species interactions; estimated associations may instead reflect missing environmental variables rather than true biotic interactions \cite{blanchet2020co}.

\noindent\textbf{Deep SDMs}

Recent advances in machine learning, particularly deep learning, have enabled improvements in biodiversity forecasts through the development of deep learning-based SDMs (deep SDMs). These models can leverage large-scale opportunistic datasets, high-resolution satellite imagery, and species interaction data to enhance predictive performance \cite{brun2024, cole2023, Deneu2021, zbinden2024on}. 
By leveraging citizen science and remote sensing, which provide critical information at fine temporal scales, deep SDMs have also been used to predict ecological community composition \cite{gillespie2024,dollinger2024,hu2025introduction}.
% 
Yet these temporally resolved forecast based on deep SDM still suffer from the same limitations as those from traditional SDMs: they assume species are in equilibrium with their environments, neglecting transient dynamics. The models implicitly treat changes as instantaneous, which overlooks important ecological processes such as population growth, dispersal, and delayed responses to environmental changes \cite{isaac2014, zurell}. This equilibrium assumption fails to capture the time-lagged nature of species range shifts and their transient responses to disturbances \cite{barber-omalley2022}. Furthermore, given the high-dimensional input predictors they can handle and the complex relationships they can capture, deepSDM are even more prone to capturing correlations between species populations and the environment which are not transferable to different contexts (e.g. distant areas, future scenarios).

\subsection{Process-based species distribution models}
Process-based models are models that explicitly represent processes such as demographic processes and dispersal \cite{cantrell2004, bonneau2016}. These models embed strong ecological priors, which theoretically make them more data-efficient and capable of extrapolating beyond observed data \cite{cabral2017, briscoe2019}. However, process-based models are often difficult to calibrate and are entailed with inaccuracies in the representation of dynamical processes, leading to imprecise forecasts in practice \cite{evans2016a, connolly2017, Scheiter2013, boussange2024}.


\section{Existing scientific machine learning approaches in related fields}

\subsection{Hard constraints}
By-design
\begin{itemize}
    \item Universal differential equations, data-driven parametrization of differential equation-based models
    \item Using causal DAGs to constraint dependencies between variables for instance via structural equation models \cite{da2024towards}, known ecological interactions \cite{poggiato2025integrating}
\end{itemize}

\subsection{Soft constraints}
Penalizing downstream the data-driven model for violating constraints.

"Physics-informed neural networks" (PINNs) are collocation methods relying on a neural network trained to predict an empirical dataset while respecting additional constraints provided by a process-based model \cite{Raissi2019, Kashinath2021,daw2021,bezenac2018}. PINNs and related paradigms, more generally coined "scientific machine learning" \cite{Rackauckas2020}, improve model generalization and data efficiency \cite{Raissi2019, Rackauckas2020}, and have been successfully applied to a variety of system with complex spatio-temporal dynamics, such as climate and weather predictions \cite{Kashinath2021, lam2023, kochkov2024}. PINNs have been extended successfully in the domain of systems biology \cite{lagergren2020, Yazdani2020}, where in contrast to the climate and weather field, process-based models are inaccurate. Despite these successes, scientific machine learning is only starting to be used in the context of biodiversity modeling \cite{boussange2024}. 



\section{Opportunities for scientific machine learning in species distribution modelling}



\noindent\textbf{Existing knowledge}
\begin{itemize}
    \item Traits: using traits to infer interactions (trait-matching \cite{pichler2020machine}), using traits to constrain species-environmental relationships \cite{pollock2012role}, using traits to parameterize species interactions, dispersal and growth rates in process-based SDMs \cite{chalmandrier2022predictions}, using community-level traits to constrain SDM predictions (Deschamps et al, in revision).  
    \item Phylogeny
    \item Physiological constraint
    \item Food webs / interactions: \cite{poggiato2025integrating}
    \item Dispersal: \cite{barber-omalley2022}: A hybrid species distribution model combining dispersal, multi-habitat suitability, and population dynamics for diadromous species under climate change scenarios, 
    \item Historical contingency
    \item Association between species and environment with GNN: \cite{harrell2025}: 
\end{itemize}

\noindent\textbf{Retrieval augmented generation.} LLMs + check by experts (human in the loop ?)


\section{Robust evaluation under distribution shifts}

\begin{itemize}
    \item With synthetic data
    \item Space for time data
    \item Invasive species
    \item Use paleo-ecological data
\end{itemize}


\section{Conclusion}

